\documentclass{sig-alternate}
\usepackage{epigraph}
\usepackage{amssymb}
\usepackage{mathtools}
\usepackage{array} 
\newcolumntype{C}{>{$}c<{$}} % math-mode version of "l" column type

\newtheorem{theorem}{Theorem}[section]
\newtheorem{lemma}[theorem]{Lemma}
\newtheorem{proposition}[theorem]{Proposition}
\newtheorem{corollary}[theorem]{Corollary}
\newtheorem{conjecture}[theorem]{Conjecture}
\newtheorem{remark}[theorem]{Remark}

\newenvironment{definition}[1][Definition]{\begin{trivlist}
\item[\hskip \labelsep {\bfseries #1}]}{\end{trivlist}}
\newenvironment{example}[1][Example]{\begin{trivlist}
\item[\hskip \labelsep {\bfseries #1}]}{\end{trivlist}}
\newenvironment{problem}[1][Problem]{\begin{trivlist}
\item[\hskip \labelsep {\bfseries #1}]}{\end{trivlist}}

\begin{document}
\title{Notes on the Analysis of Boolean Functions}

\numberofauthors{2}
\author{
 % 1st. author
 \alignauthor Ravi Kant \\
 \affaddr{Yahoo Labs, Sunnyvale}\\
 \email{rkant@yahoo-inc.com}
}

\maketitle

\section{Components and Complements}
Let $f:\mathbb{F}_2^n \rightarrow \mathbb{R}$ and let its Fourier expansion be
\[ f(x) = \sum_{S\subseteq [n]}\hat{f}(S)\chi_S(x) \]
\begin{definition}
For a subset $I\subseteq [n]$ we define the $I${\em -component}, $f_I$ of $f$ as
\[ f_I(x) = \sum_{S\subseteq I}\hat{f}_S\chi_S(x) \]
and the $I${\em -complement}, $\bar{f}_I$ as
\[ \bar{f}_I(x) = \sum_{S\nsubseteq I}\hat{f}_S\chi_S(x) = f(x) - f_I(x) \]
\end{definition}

\begin{remark}
\[ \sum_{x\in\mathbb{F}_2^n}f(x) = \sum_{x\in\mathbb{F}_2^n}f_I(x)\] since $\hat{f}(\varnothing) = \hat{f}_I(\varnothing)$
\end{remark}

We will interpret a vector $x \in \mathbb{F}_2^n$ as the indicator of a subset $S\subseteq [n]$ and (abuse of notation) use the same letter for the vector and the set when the interpretation is clear from the context.

\begin{lemma}
 Let 
 \[\delta_I(x) = \begin{cases} 2^{|I|} & \text{if } I\cap x = \varnothing \\ 0 & \text{otherwise}\end{cases} \]
then,
\[ f_I = f * \delta \]
\end{lemma}

\begin{proof}
\begin{eqnarray*}
   f_I(x) &=& \sum_{S\subseteq I}\hat{f}_S\chi_S(x) \\
            &=& \sum_{S\subseteq I}\chi_S(x)\frac{1}{2^n}\sum_{y\in \mathbb{F}_2^n}f(y)\chi_S(y) \\
            &=& \frac{1}{2^n}\sum_yf(y)\sum_{S\subseteq I} \chi_S(x+y) \\
            &=& \frac{1}{2^n}\sum_yf(y)\delta_I(x+y)
\end{eqnarray*}
\end{proof}

\begin{corollary}
If $f:\mathbb{F}_2^n \rightarrow \{0,1\}$ then \[\forall x,I,\mbox{  } 0\leq f_I(x) \leq 1\]
\end{corollary}
\begin{proof}
$\delta_I$ is a probability density function (i.e. $\delta_I/2^n$ is a probability distribution.)
\end{proof}

\begin{theorem}
(Bourgain) Let $f:\mathbb{F}_2^n\rightarrow\{0,1\}$. Fix a subset $I\subseteq [n]$. Then,
\[ 2\sum_{x\in\mathbb{F}_2^n}\bar{f}_I(x)^2 =   \sum_{x\in\mathbb{F}_2^n}|\bar{f}_I(x)| \]
\end{theorem}

\begin{proof}
\begin{eqnarray*}
LHS &=& 2\sum_{x\in\mathbb{F}_2^n}(f(x) - f_I(x))^2 \\
        &=& 2f\cdot f + 2f_I\cdot f_I - 4f\cdot f_I  \\
        &=& 2f\cdot f - 2f\cdot f_I \mbox{  (since }f_I \cdot f_I = f \cdot f_I \mbox{)}  \\
 RHS &=& \sum_{x\in\mathbb{F}_2^n}|f(x) - f_I(x)|  \\
 	 &=& \sum_{f(x) = 0} f_I(x) + \sum_{f(x) = 1} (f(x) - f_I(x))  \mbox{ (see corollary)}  \\
	 &=&  \sum_{x\in\mathbb{F}_2^n}f(x) + \sum_{x\in\mathbb{F}_2^n}(1-2f(x))f_I(x)  \\
	 &=&  f\cdot f + f\cdot f - 2f\cdot f_I 
\end{eqnarray*}
\end{proof}
\newpage
\section{Low Degrees}
\begin{problem}
Characterize all Boolean functions with degree $\leq 1$.
\end{problem}

\begin{lemma}
The only degree $\leq 1$ boolean functions are $\pm \chi_S$ where $|S| \leq 1$. 
\end{lemma}

\begin{proof}[1]
$f = \sum_{i=0}^{n} \hat{f}_i\chi_i$ where $i$ refers to the empty set when $i=0$ and the singleton set $\{i\}$ otherwise and let $x^{(i)}$ denote its vector indicator. Since $f$ is boolean,
\begin{eqnarray*} 
f(1) & = & \pm 1 = \sum_{0}^{n} \hat{f}_i \\
f(x^{(i)}) & = & \pm 1 = \hat{f}_0 + \cdots \hat{f}_{i-1} - \hat{f}_i + \hat{f}_{i+1} + \cdots + \hat{f}_n \\
\implies \hat{f}_i & = & 0 \mbox{ or } \pm 1
\end{eqnarray*}
Also we know that $\sum_{0}^{n} \hat{f}_i^2 = 1$. QED.
\end{proof}

\begin{proof}[1 Rehashed]
\begin{eqnarray*}
f & = & \sum_{i=0}^{n} \hat{f}_i\chi_i \\
\implies D_i[f] & = & \hat{f}_i\chi_0 \mbox{   (where } 1\leq i \leq n \mbox{)} \\
\implies \forall x, D_i[f](x) & = & \hat{f}_i \\
	& = & 0 \mbox{ or } \pm 1 \mbox{ (since $f$ is boolean)}
\end{eqnarray*}
As in first proof, QED
\end{proof}
 
\section{Correlated Gaussians}
\begin{lemma}
Let $N(\mu,\sigma^2)$ be the normal distribution with mean $\mu$ and variance $\sigma^2$. Then \[N(\mu_1,\sigma_1^2) + N(\mu_2, \sigma_2^2) = N(\mu_1 + \mu_2, \sigma_1^2 + \sigma_2^2)\] where the LHS random variables are assumed to be independent.
\end{lemma}

\begin{lemma}
If $G = H = N(0,1)$ and $E[GH] = \rho$ then $H = \rho G + \sqrt{1 - \rho^2}G'$ where $E[G,G'] = 0$
\end{lemma}
\begin{proof}
Define $G' = \frac{1}{\sqrt{1 - \rho^2}}(H - \rho G)$. Using the preceding lemma, $G' = N(0,1)$. Using $E[GH] = \rho$,
\begin{eqnarray*}
E[GG'] & =  & \frac{1}{\sqrt{1 - \rho^2}}E(GH - \rho G^2) \\
 & = & \frac{\rho - \rho \cdot 1}{\sqrt{1 - \rho^2}} \\
 & = & 0
\end{eqnarray*}
\end{proof}

\begin{lemma}
Let G,H be standard Gaussians with $E[GH] = \rho$. Then $P[sgn(G) \neq sgn(H)] = \cos^{-1}\frac{\rho}{\pi}$.
\end{lemma}

\begin{proof}
Consider the joint distribution $[G,G']$ from previous lemma. We need to compute the relative area of the region $R := \{(x,y): sgn(x) \neq sgn(\rho x + \sqrt{1 - \rho^2}y)$. The result follows from the circular symmetry of the joint distribution.
\end{proof}

This leads to the following:
\begin{itemize}
\item $W^k(Maj_n) \sim (\frac{2}{k\pi})^{3/2}$, when $k$ is odd, $0$ o/w.
\item $W^{>=k}(Maj_n) \sim (\frac{2}{\pi})^{3/2} \cdot k^{-1/2}$
\end{itemize}

\section{Some special functions}
\noindent We will use the following notation: For any subset $S \subseteq [n]$
\begin{itemize}
\item $e_S$ and $\chi_S$ are functions on the boolean cube
\item $e'_S$ and $\chi'_S$ are functions on the dual boolean cube.
\end{itemize}

\subsection{Spectra}
\subsubsection {Neighbourhoods of a character}
\[ min_n = OR_n = \begin{bmatrix}
				1 \\
				-1 \\
				\vdots \\
				-1
				\end{bmatrix}
				 =
				 \begin{bmatrix}
				-1 \\
				-1 \\
				\vdots \\
				-1
				\end{bmatrix}
				+
				\begin{bmatrix}
				2 \\
				0 \\
				\vdots \\
				0
				\end{bmatrix}
				\]
\[ \widehat{OR}_n = \begin{bmatrix}
				-1 + \frac{2}{2^n}\\
				\frac{2}{2^n} \\
				\vdots \\
				\frac{2}{2^n}
				\end{bmatrix}
				= -e'_{\varnothing} + \frac{2}{2^n}\chi'_{\varnothing}
				\]
Similarly,
\[ \widehat{AND}_n = \begin{bmatrix}
				1 - \frac{2}{2^n}\\
				\vdots \\
				\frac{2(-1)^{|S| + 1}}{2^n} \\
				\vdots \\
				\end{bmatrix}
				= e'_{\varnothing} - \frac{2}{2^n}\chi'_{[n]}			
				\]

In general,				
\[ 1_{S}: \mathbb{F}_2^n \rightarrow \{-1,1\} \] where $1_S(a) = -1$ iff $a$ is the indicator of $S$, then
\[ \hat{1}_{S} = e'_{\varnothing} - \frac{2}{2^n}\chi'_S \]

Obvious generalizations:
\begin{itemize}
\item $EQ_n = 1 - 2e_{\varnothing} - 2e_{[n]} \implies \widehat{EQ}_n = e'_{\varnothing} - \frac{2}{2^n}\chi'_{\varnothing} - \frac{2}{2^n}\chi'_{[n]} $
\item $ \widehat{NAE}_n = -e'_{\varnothing} + \frac{2}{2^n}\chi'_{\varnothing} + \frac{2}{2^n}\chi'_{[n]} $
\item Analogous results for the neighbourhoods of a non-trivial character
\end{itemize}

\subsubsection{Quadratic functions}
\noindent For $x_1, x_2 \in \mathbb{F}^n_2$ we define \[IP_2(x_1, x_2) = (-1)^{x_1\cdot x_2}\]
Clearly:
\begin{eqnarray*}
\langle IP_2, \chi_{S_1, S_2}\rangle & = & \frac{1}{2^{2n}}\sum_{x_1, x_2} (-1)^{x_1\cdot x_2}(-1)^{x_1\cdot S_1}(-1)^{x_2\cdot S_2} \\
& = & \frac{1}{2^{2n}} \sum_{x_1}(-1)^{x_1\cdot S_1} \sum_{x_2}(-1)^{(x_1+S_2)\cdot x_2} \\
& = &  \frac{1}{2^{2n}} \sum_{x_1}(-1)^{x_1\cdot S_1}\cdot 2^n\delta(x_1,S_2) \\
& = &  \frac{1}{2^n} (-1)^{S_2\cdot S_1}
\end{eqnarray*}

Note that this shows $IP_2$ is self dual. \\

\subsubsection{The Complete Quadratic and Variants}

For $x = (x_1,\ldots, x_n) \in \mathbb{F}^n_2$ we define \[CQ(x) = (-1)^{\sum_{i<j}x_ix_j}\]
\begin{eqnarray*}
\langle CQ, \chi_S\rangle & = & \frac{1}{2^n}\sum_x (-1)^{S\cdot x}\cdot (-1)^{|x| \choose 2} \\
& = & \frac{1}{2^n}\big[\sum_{k}(-1)^k\big(\sum_{\substack{|x| = 0(4)\\S\cdot x = k}}1 + \\
& &	\sum_{\substack{|x|=1(4)\\S\cdot x = k}}1 - \sum_{\substack{|x|=2(4)\\S\cdot x = k}}1 -\sum_{\substack{|x|=3(4)\\S\cdot x = k}}1\big)\big] \\
& = & \frac{1}{2^n}[S_0 + S_1 - S_2 -S_3] \\
& = & \hat{f}_S (say)
\end{eqnarray*}

\noindent The second line in the above equations comes from the fact that ${n \choose 2}$ is even if $n$ is 0 or 1 mod 4 and odd otherwise.\\

\noindent Now let $\sigma_0(n) = \sum_{i=0(4)} {n \choose i}$, similarly $\sigma_k(n) = \sum_{i=k(4)} {n \choose i}$.  We know that:
\[ 4\sigma_k(n) = 2^n + 2^{\frac{n}{2} + 1}\cos((n-2k)\frac{\pi}{4})\]
Also let $|S| = d$. We now evaluate the above 4 sums in turn:
\begin{eqnarray*}
S_0 & = & \sum^{k = d}_{S\cdot x = k = 0}(-1)^k\sum_{|x| = 0(4)}1 \\
& = & \sigma_0(n-d) + (-1) {d \choose 1}\sigma_3(n-d) + \cdots \\
& & + (-1)^k{d \choose k} \sigma_{4-k}(n-d) + \cdots \\
& = & \sigma_0(n-d)\sigma_0(d) - \sigma_1(n-d)\sigma_3(d) + \\
& & \sigma_2(n-d)\sigma_2(d) - \sigma_3(n-d)\sigma_1(d)\\
\\
Similarly,\\
S_1 & = &  -\sigma_0(n-d)\sigma_1(d) + \sigma_1(n-d)\sigma_0(d) + \\
& & -\sigma_2(n-d)\sigma_3(d) + \sigma_3(n-d)\sigma_2(d)\\
\\
S_2 & = &  \sigma_0(n-d)\sigma_2(d) - \sigma_1(n-d)\sigma_1(d) + \\
& & \sigma_2(n-d)\sigma_0(d) - \sigma_3(n-d)\sigma_3(d)\\
\\
S_3 & = &  -\sigma_0(n-d)\sigma_3(d) + \sigma_1(n-d)\sigma_2(d) + \\
& & -\sigma_2(n-d)\sigma_1(d) + \sigma_3(n-d)\sigma_0(d)
\end{eqnarray*}

Substituting for the $\sigma_i$
\[ \hat{f}_S = \sqrt{\frac{2}{2^n}}\sin((n-2d+1)\frac{\pi}{4}) \]

For even $n$ the $\sin$ term is $\pm\frac{1}{\sqrt{2}}$ so CQ is bent.

\begin{remark}
Since $CQ$ and $\widehat{CQ}$ are symmetric functions we can talk of $CQ(d)$ and $\widehat{CQ}(d)$ where $d$ is the degree of the corresponding inputs. When $n$ is even we have
\[ sgn(\widehat{CQ}(d)) = CQ((d - \frac{n}{2} + 1) mod 4)\]
\begin{center}
\begin{tabular}{| C | C | C | C | C | C |}
\hline
& & \multicolumn{4}{C|}{\bf sgn(\widehat{CQ}(d))} \\ 
\hline
{\bf d (mod 4)} & {\bf CQ(d)} & n \equiv 0(8) & n \equiv 2(8) & n \equiv 4(8) & n \equiv 6(8) \\
\hline
0 & +1 & +1 & +1 & -1 & -1 \\
1 & +1 & -1 & +1 & +1 & -1 \\
2 & -1 & -1 & -1 & +1 & +1 \\
3 & -1 & +1 & -1 & -1 & +1 \\
\hline
\end{tabular}
\end{center} 

\end{remark}

\begin{lemma}
Let $n$ be even and let $CQ$ be the complete quadratic function on $n$ variables. Let $S$ be an arbitrary subset of $[n]$ and let $|S| = s$.
\begin{enumerate}
\item If $(s - \frac{n}{2} + 1) \equiv 0(4))$ then $CQ\cdot \chi_S$ is self-dual bent.
\item If $(s - \frac{n}{2} + 1) \equiv 2(4))$ then $CQ\cdot \chi_S$ is anti-self-dual bent.
\end{enumerate}
\begin{center}
\begin{tabular}{| C | c | c | c | c |}
\hline
& \multicolumn{4}{C|}{\bf |S| \mod 4} \\ 
\hline
{\bf \frac{n}{2}(mod 4)} & {\bf 0} & {\bf 1} & {\bf 2} & {\bf 3} \\
\hline
0 & X & anti & X & dual \\
\hline
1 & dual & X & anti & X \\
\hline
2 & X & dual & X & anti \\
\hline
3 & anti & X & dual & X \\
\hline
\end{tabular}
\end{center} 
\end{lemma}
\begin{proof}
We have already seen that when $n$ is even, $CQ$ is bent and consequently $CQ\cdot \chi_S$ is bent for any $S$. \\

Let $g(x) = CQ(x)\chi(x)$. Let $x$ be an arbitrary vertex on the boolean cube and $\alpha$ the corresponding vertex on the dual boolean cube. Given that $g$ is bent, we just need to show that $g(x)$ and $\hat{g}(\alpha)$ have the same(resp. opposite) sign for all $x$ to conclude that $g$ is self dual (resp. anti-self-dual). \\

Let $|x| = |\alpha| = d$ and $S\cdot x = c$.
\begin{eqnarray*}
sign(\hat{g}(\alpha)) & = & sign(\widehat{CQ}(\alpha\oplus S)) \\
& = & sign(\widehat{CQ}(d + s - 2c) \\
& = & sign(CQ((d + s - 2c -\frac{n}{2} + 1) mod 4) \\
& = & (-1)^csign(CQ((d + s - \frac{n}{2} + 1) mod 4) \\
& = & (-1)^{\frac{s - \frac{n}{2} + 1}{2}}\cdot (-1)^c \cdot sign(CQ(d)) \\
& = & (-1)^{\frac{s - \frac{n}{2} + 1}{2}}\cdot g(x)
\end{eqnarray*}

QED.
\end{proof}
\newpage

\section{Degree, Influence, Juntas}
\noindent In this section we only consider functions $f:\{-1, 1\}^n \rightarrow \mathbb{R}$.
For a random variable $X$ (e.g. $f(x)$ where $x \sim \{-1, 1\}^n$) we define, $\|X\|_p = \mathbb{E}[|X|^p]^{\frac{1}{p}}$ and analogously $\|f\|_p$.
\begin{theorem}
If $deg(f) = k$ then $f$ is a $k2^{k-1}$-junta. Note that a depth-$k$ decision tree is a $2^{k-1}$-junta, so the bound is tight up to the linear factor.
\end{theorem}
\begin{proof}
At most $k2^{k-1}$ variables have non-zero influence:
\begin{enumerate}
\item For a degree-$k$ function \begin{eqnarray*} I_i[f] \geq \frac{1}{2^{k-1}} & or & I_i[f] = 0 \end{eqnarray*}
	\begin{enumerate}
		\item $I_i[f] = \mathbb{P}[D_i[f] \neq 0]$
		\item $deg(D_i[f]) = k - 1$ 
		\item If $p:\{-1,1\}^n \rightarrow \mathbb{R}$ is a degree-$k$ polynomial then \begin{eqnarray*} p(x) \cong 0 & or & \mathbb{P}[p(x) \neq 0] \geq 2^{-k} \end{eqnarray*}
		\begin{enumerate}
			\item Induction on $n,k$.
			\item Let $p(x_1,\ldots,x_n, \pm 1) = p_{\pm} = q(x) \pm r(x)$, where $r(x)$ has degree $k-1$.
			\item If $p_+ \ncong 0$ and $p_- \ncong 0$, done.
			\item If $p_+ \cong 0$ then $p_-$ has degree $\leq k - 1$, done.
		\end{enumerate}
	\end{enumerate}
\item $I[f] \leq deg(f) = k$. 
\end{enumerate}
\end{proof}

Notes:
\begin{enumerate}
\item \textbf{\textit{Bonami's Lemma}}: degree-$k$ functions are $9^k$-reasonable.\begin{proof}
$ deg(f) = k \implies \|f\|_q \leq \sqrt{q-1}\|f\|_2 $
\end{proof}
\item \textit{\textbf{Paley-Zygmund inequality}}: For $X \geq 0$ and $0\leq t \leq 1$, $\mathbb{P}(X \geq t\mu) \geq (1-t)^2\frac{\mu^2}{\mathbb{E}[X^2]}$
\begin{proof}
\[\mu = \mathbb{E}[X] = \sum_{x < t\mu}x\mathbb{P}(x) + \sum_{x\geq t\mu}\sqrt{x^2\mathbb{P}(x)}
\sqrt{\mathbb{P}(x)}\]
Using Cauchy-Schwartz on second term,
\[\mu \leq t\mu + \sqrt{\mathbb{E}[X^2]}\sqrt{\mathbb{P}[X > t\mu]} \]
\end{proof}
\item\textit{\textbf{Tail lower bounds}}
      \begin{enumerate}
 	\item \textit{\textbf{B-reasonable variables}}: If $X$ is B-reasonable,
\[\mathbb{P}[|X| \geq t\|X\|_2] \geq \frac{(1-t^2)^2}{B} \]
\begin{proof}
Paley-Zygmund applied to $X^2$.
\end{proof}
	\item \textit{\textbf{Degree-$k$ functions}}:
\[ \Pr_{x \sim \{-1, 1\}^n}[|f(x) - \mu_f| \geq t\sigma_f]  \geq \frac{(1 - t^2)^2}{9^k} \]
	\begin{proof}
	Set $X = \frac{f - \mu}{\sigma}$. Clearly $\|X\|_2 = 1$ and $X$ is $9^k$-reasonable.
	\end{proof}
	\end{enumerate}
\end{enumerate}

\subsection{\textbf{W$^1$[f]}}

\begin{theorem}({\bf FKN})
If $f$ is $\epsilon$-concentrated \underline{\textbf{at}} degree $1$ then $f$ is $f$ is $O(\epsilon)$-close to $\pm\chi_i$ for some $i \in [n]$.
\end{theorem}
\begin{proof}
Let $l = f^{=1}$ and $\hat{f}_{max} = \max_{1 \leq i \leq n}|\hat{f}_i|$.
\begin{enumerate} 
\item It is enough to show that $Var[l^2] = O(\epsilon)$.
	\begin{eqnarray*}
	l^2 & = & (1-\epsilon) + 2\sum_{i < j}\hat{f}_i\hat{f}_j\chi_{i,j} \\
	\implies Var[l^2] & = & 4\sum_{i < j}\hat{f}^2_i\hat{f}^2_j \\
	\frac{1}{2}Var[l^2] & = & \sum_{i \neq j}\hat{f}^2_i\hat{f}^2_j \\
	& = & (\sum_i \hat{f}^2_i)^2 - \sum_i\hat{f}^4_i \\
	\implies \sum_i\hat{f}^4_i & = & (1-\epsilon)^2 - \frac{1}{2}Var[l^2] \\
	\hat{f}_{max}^2(\sum_i \hat{f}^2_i) & \geq & (1 - \epsilon)^2 - \frac{1}{2}Var[l^2] \\
	\implies \hat{f}_{max} & \geq & (1 - \epsilon) - \frac{Var[l^2]}{2(1 - \epsilon)}
	\end{eqnarray*}
\item If $l^2$ is bounded away from 1, then $l$ is bounded away from $f$: let $\epsilon < 10^{-4}$ and $t = \frac{1}{2}$. Suppose $Var[l^2] > 2601\epsilon$, \begin{eqnarray*}
	\mathbb{P}[|l^2 - 1 + \epsilon| \geq \frac{51}{2}\sqrt{\epsilon}] & \geq & \frac{1}{144}  \mbox{ (using 3b)}\\
	\implies \mathbb{P}[|l^2 - 1| \geq \frac{50}{2}\sqrt{\epsilon}] & \geq & \frac{1}{144} \\
	\implies \mathbb{P}[(f - l)^2 > 146\epsilon] & \geq & \frac{1}{144} \\
	\implies \mathbb{E}[(f-l)^2] & \geq & \frac{146}{144}\epsilon \\
	\mbox{But } \mathbb{E}[(f-l)^2] & = & \epsilon
	\end{eqnarray*}
\end{enumerate}
\end{proof}

\begin{lemma}
\textit{\textbf{deg $\leq$ 1}}: If $f$ is $\epsilon$-concentrated \underline{\textbf{up to}} degree 1 then $f$ is $O(\epsilon)$-close to a 1-junta ($k$-junta depends on at most $k$ variables).
\end{lemma}
\begin{proof} Apply FKN to $g$, defined as:
\[g(x_0,x) = x_0f(x_0x) = \hat{f}_{\varnothing}x_0 + \sum_{|S| = 1}\hat{f}_Sx^{S} + \sum_{|S| = 2}\hat{f}_Sx_0x^S + \ldots\]
\end{proof}
\begin{theorem}
Let $f:\{-1,1\}^n \rightarrow \{0,1\}$ and $\mathbb{E}[f] = \alpha < \frac{1}{2}$ (otherwise consider $1 - f$). 
\[W^1[f] \leq 2\alpha^2\log(\frac{1}{\alpha})\]
\end{theorem}




\bibliographystyle{abbrv}
\bibliography{influence}
\end{document}
